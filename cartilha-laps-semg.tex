% cartilha-laps-semg.tex
%

\documentclass[a4paper,11pt]{article}

%=== preambulo === 
\usepackage[brazilian]{babel}
\usepackage{color,enumerate}
\usepackage[brazilian]{babel}
\usepackage{color,enumerate}
\usepackage{graphicx}
\usepackage[toc,page]{appendix}
\usepackage[table]{xcolor}
\usepackage{booktabs}
\usepackage[T1]{fontenc}
\usepackage[utf8]{inputenc}
\usepackage{subfigure}
\usepackage[backend=biber, style=authoryear, doi=false, isbn=false, url=false]{biblatex}
\addbibresource{bib/semg.bib}

%\usepackage[backref=true, pdfborder= {0 0 0}, citecolor=black, urlcolor=black, linkcolor=black, colorlinks=true]{hyperref}
%\usepackage[final]{pdfpages}
%\definecolor{lightgray}{gray}{0.9}
%\definecolor{lightblue}{RGB}{153,204,255}
%\definecolor{lightgreen}{RGB}{204,255,229}
%%\definecolor{lightyellow}{RGB}{255,255,153}
%\definecolor{lgray}{gray}{0.7}
% === ===

\title{Aquisição de sinais de eletromiografia de superfície\footnote{Este material é baseado no Trabalho de Conclusão de Curso ``Desenvolvimento e análise de protótpos para aquisição de sinal mioelétrico de superfície'', de autoria de Sérgio de Nazaré Rodrigues Lima Júnior em agosto de 2019, como requisito para obtenção do grau de Engenheiro de Telecomunicações.}}
       
\author{Grupo de análise de movimento do LaPS\\
        (LaPS/ITEC/UFPA)}
\date{Março de 2020} 


\begin{document}
  \maketitle
  \section{Introdução}
  \label{sec:intro}
  Uma das formas mais utilizadas para diagnóstico e tratamento de pacientes com
problemas musculares é o uso da eletromiografia. A eletromiografia é muito solicitada por várias áreas da saúde devido ser um procedimento simples e seguro, que ajuda a diagnosticar problemas comuns como formigamentos, fraquezas musculares, dores e cãibras. Também é extremamente eficaz para a identificação de doenças mais sérias, que afetam as células nervosas ou os nervos periféricos (MIOTEC, 2019). Inumeras são as áreas onde a eletromiografia é utilizada:
   \begin{itemize}
     \item Na medicina, identificando disfunções musculares de pacientes além de diagnosticar possíveis doenças graves que atigem os nervos e músculos;
     \item Na odontologia, trabalhando a região muscular da articulação mandibular;
     \item Na fonoaudiologia, com a avaliação do tratamento dos músculos pertencentes ao processo da fala;
     \item Na fisioterapia com a avaliação do potencial de resposta das celulas musculares;
     \item Na educação física, avaliando as condições dos músculos, prevenindo lesões e aplicando programa de exercícios mais adequado.
     \item No desenvolvimento de próteses interativas de membros (mãos, braços e pernas), sistemas de locomoção (cadeira de rodas), etc.
   \end{itemize}

  \section{Características dos sinais de SEMG}
  \label{sec:carac}
 
  \section{Sistema de aquisição desenvolvido}
  \label{sec:sist}
  
  \section{Conclusão}
  \label{sec:concl}
 
  \printbibliography
\end{document}
